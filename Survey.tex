\documentclass[11pt]{article}
\usepackage{cite}

\begin{document}

% Subject : Symbolic executions technique for finding bugs %
\title{A Survey of\\Symbolic Executions Techniques} % NB: I removed the "for finding bugs" because "bug" is not well-defined and we can clearly state in the introduction why we use symbolic execution.
\author{Hallet Adrien \and Sens Loan}
\date{\today}
\maketitle

  \section*{Abstract}
    % Describe paper's goals and content

  \section{Introduction}
    \subsection{(Attempting) A definition}
      % I tried to explain what is symbolic execution in simple concepts of software engineering
      The first occurences of symbolic execution described the then-new method as a middle ground~\cite{newapproach} between the two most-used method of its time. On one hand, program testing (\emph{e.g.: unit testing}) can not always detect a fault in a program and producing a correct test sample and proving that it indeed is correct is not that easy. On the other hand, program proving can indeed ensure that a program is correct from its entry point to the result but it heavily relys on the proof definitions by the programmer and the formal definition of the problem.\\
      Nowadays, symbolic execution is both described as (part of) the core of many modern techniques to software testing\cite{chopper:icse18} and an effective way to create tests suites with extensive coverage.\cite{threedecadeslater}
    \subsection{The concept}
      % TODO: Adrien <= I'm working on this to warm up :)
    % What is symbolic execution and why do we do this survey

  \section{History}
    % Draft the history of symbolic execution, dig any trend up or whatever we can say about it.

  \section{Methods}
    % Begin with anything general to say (if any) then detail methods, could also be useful to compare them (if possible) and/or to say when/why a particular method is being used instead of another.
    % Maybe divide the general techniques and optimisations we can apply to all of them ?
    
    \subsection{Concolic execution}
    	The name "concolic" is a portmanteau of the words "concrete" and "symbolic", the idea of this testing method is to mix symbolic execution alongside concrete ones.\\
    	
    	This technique concept was first introduced on 2005 \cite{godefroid2005dart}. Since then the idea was further extended and combined with other testing techniques.

  \section{Tools and languages}
    % Anything general to say bout the content (maybe explain the omnipresence of Microsoft in the market)
    % List tools and language, compare them if possible

  \section{Conclusions}
    % What can we get from this paper in general
    % Could be pertinent to talk about a kind of "to go further" which could contain adjacent fields of research
    % Could be pertinent to talk about the future of symbolic execution, if there is any new technique being developed, ...

\bibliography{bibliography}{}
\bibliographystyle{plain}
\end{document}
